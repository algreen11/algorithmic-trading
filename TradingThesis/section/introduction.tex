\documentclass[../thesis.tex]{subfiles}

\begin{document}
\chapter{Introduction}
\label{ch:intro}

\textit{Algorithmic Trading} (AT) is the generation and submission of orders of a financial asset by an algorithm, or set of instructions \cite{Treleaven2013}. The trading of stocks is one of the most important and prominent profit making utilities in the world. The introduction of 401k's in 1978 into employment benefits has caused the majority of families in the United States to be invested in the stock market \cite{Stobierski2018}. Each individual's portfolio has a unique distribution of stocks. Because the average person doesn't have enough knowledge to day trade productively, this has created the need for owning stocks that provide safe and constant returns. This is realized in mutual funds, which are collections of funds invested into a variety of different stocks and often compose large percentages of 401k portfolios. These funds often utilize algorithmic trading to produce maximum profit, motivating our study to understand how these strategies work. This has allowed millions of people to maximize longterm savings \cite{Treleaven2013}.

To enter into the stock market one must purchase a specified number of shares of a particular security or mutual fund, which consist of companies, individuals, institutions, currencies, and more. However, this is a two-way transaction. To purchase a specific number of shares at a particular price, another party must agree to sell at least as many shares at that same price. This happens via broker or in modern times, via computer. From the trader's standpoint, it is as simple as logging onto a stock trading website and clicking "execute trade". 

There are many different types of transactions in financial markets. \textit{Buying} and \textit{selling} of shares are among the most common stock orders. \textit {Limit orders} are buy and sell transactions that are only executed at a given limit price threshold. These orders are generally used to minimize loss and risk as trades will only occur when specific market conditions hold \cite{Aldridge2010}. \textit{Shorting} of stocks works the opposite way of buying. With shorting, the trader is betting against the performance of the stock. The short sellers borrow shares of stock they don't own and sell them at market price. The goal is to re-buy the stock at a later date and return the borrowed shares to the lender by repurchasing the stocks at a lower price than the initial purchase. \textit {Options} trading has gained recent popularity and is comprised of call and put options. This is a derivative type of security, as it is intrinsically linked to the price of something else. \textit {Call options} give the owners the right to buy stock at a certain price, while \textit{put options} gives the holder the right to sell stock at a certain price. Here, we are chiefly concerned with buying and selling transactions, however further study could examine some of the other types of orders mentioned above. 

 The intended outcome for any trade or transaction is always the same: to maximize profit. However, it is often difficult to pinpoint the exact moment in time that a particular trade maximizes profit. As computers and network connections have improved, trading financial instruments via automation has become more prominent. By using advanced mathematical models and measures, automated financial trading aims to maximize profit, executing up to thousands of trades a day of a particular security. While not often widely publicized, millions of trades each day are made by computer algorithms and not humans. In 2011, over 73\% of all equity trading volume in the U.S. was performed algorithmically \cite{Treleaven2013}. However, much of this trading is done by large financial institutions or prop-trading firms. Few studies look at the efficacy and inner workings of prominent trading algorithms \cite{Ehlers} \cite{Gatev2006}. 
  
The stock market at its inception was entirely analog and trades of stock were carried out in person. Rapid trading traces its roots back to the early 1930s. Specialists and pit traders bought and sold positions and broadcasted trades via new high speed telegram services \cite{Treleaven2013}.  Computerization of trades started in the 1980s when the NASDAQ was the first exchange to introduce purely electronic trading. This trend of stock markets moving towards computers opened the gate for the use of  Today, trading time has changed from a matter of seconds to microseconds \cite{Treleaven2013}. The stock market moving entirely electronic gives motivation and reason for automated trading. 

Just like the stock market moving entirely electronic, modern news has also moved entirely online. Social media or news websites are often the most effective way for a company to create news in the modern day. This allows companies to share news in the matter of seconds. Social media websites like Twitter or Facebook allow companies to create incredible amounts of content and instantly share it with the entire world. Previous research has looked to leverage news sentiment for algorithmic trading strategies, as it has been proved that news affects stock price\cite{Ho2016}. Behavioral finance suggests that emotions, mood, and sentiments in response to news play a significant role in investment and affect the price of stocks. 
 
Algorithmic trading has not always had a positive impact on the market. There have been some negative consequences. The May 6th, 2010 ``Flash Crash'' brought the public's attention to the little publicized, but very heavily used algorithmic trading in financial markets \cite{Kirilenko2017}. This happened with E-mini, denoted by ES, which is a stock market index futures contract that trades for around 50 times the value of the S\&P. A mutual fund complex sold 75,000 of these contracts valued at approximately $\$4.1$ billion - resulting in the largest net change in daily position of any trader in the E-mini since the beginning of the year. This caused a cascading effect, as other traders reacted to this massive momentary plunge and sold accordingly, with over 20,000 trades across 300 separate securities executing at prices 60 percent away from their initial prices a mere half hour earlier. The Dow Jones Industrial Average fell over 1,000 points in a matter of moments, causing over \$1 trillion to evaporate within 10 minutes. 


\section{Motivation}

AT is the generation and submission of orders of a financial asset by an algorithm, or set of instructions, that processes current market data and places orders in stock marketplaces without human interaction. More formally, Chaboud et al. \cite{Chaboud2009} define AT as: \begin{displayquote} ``In algorithmic trading (AT), computers directly interface with trading platforms, placing orders without immediate human intervention. The computers observe market data and possibly other information at very high frequency, and, based on a built-in algorithm, send back trading instructions, often within milliseconds. A variety of algorithms are used: for example, some look for arbitrage opportunities,
including small discrepancies in the exchange rates between three currencies; some
seek optimal execution of large orders at the minimum cost; and some seek to
implement longer-term trading strategies in search of profits.'' \end{displayquote}

\textit{High Frequency Trading} (HFT), is a subset of algorithmic trading and a far newer phenomenon that has been made possible by the rapid improvement of computerized trading speed \cite{Gomber2011}. This is the primary form of algorithmic trading found in financial markets today, with billions of dollars constantly traded by machines every second. However, it is difficult for trading of this type to occur at an individual level. Companies have the capital and resources to pay millions of dollars for their own fiber-optic cable connections across the United States and into Wall Street and up to the millisecond stock data whereas the individual doesn't. 

This provides an interesting conundrum for the individual trader. Much of AT done in the market is done by large banks or massive prop-trading firms trading billions of dollars of existing capital per day. Is it possible for an individual with much less capital to engage in AT and be significantly profitable? With the current market trend of no-commission brokers such as Robinhood or developer friendly API's such as Alpaca, it has become ever the more possible for individuals to become algorithmic-traders \cite{Alpaca}. However, what strategies can be implemented to obtain profits? Is it feasible to use small amounts of capital to have significant returns? While sentiment plays a role in the stock market, is it possible to quantify via Twitter and build a model that predicts stock price? All of these questions provide motivation for this study. In particular, the last question is of utmost importance. Social networks have provided large companies a platform for mass public communication with the entire planet. This data is publicly available and could provide helpful insight into trading decisions. 


\section{Our Contribution}
Because AT is an unsurprisingly secretive industry, we look to examine the effectiveness of both typical and atypical trading strategies. We give an in-depth analysis of a variety of both AT and HFT strategies. Specifically, we examine momentum, arbitrage and mean reversion measures and methods \cite{Aldridge2010}. While many of these methods can be applied to other financial markets, such as Bitcoin markets, we use data exclusively from the US stock market. We develop a test suite that runs a variety of different algorithmic trading strategies on two different periods of data. Some of the strategies tested focus on HFT and use intra-day, minute-by-minute stock data. Others look at a period of over 7 years of closing price data. By using different types of data, we are able to examine strategies in both an AT and a HFT context. 

Additionally, we contribute unique trading algorithms using both Machine and Deep Learning using Twitter \textit{sentiment} data. We collect tweets from publicly traded companies twitter pages from the inception of their account until 2017. We then using Natural Language Processing (NLP) to construct average sentiment values for tweets on each day. While we find weak linear correlation with price and average tweet sentiment, we expect more advanced models that use machine learning to be able to more accurately apply this data. We investigate the effectiveness and fit of machine learning classifiers and reggressors, such as decision trees, $k$-nearest neighbors, and MLP, and random forest. We also investigate the effectiveness of LSTM neural networks using deep learning. 

\section{Organization}
In Chapter~\ref{ch:prelim} we give\ldots, in Chapter~\ref{ch:specific1}\ldots

\added[id=AT, remark = {ToDo}] {THIS NEEDS TO BE FILLED IN ONCE STRUCTURE IS DONE}

\end{document}

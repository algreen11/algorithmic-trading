\documentclass[../thesis.tex]{subfiles}
\begin{document}
\chapter{Related Work}
\label{ch:relatedwork}

Many researchers have studied AT and HFT in the context of specific algorithms, the state of the industry, and greater topics \cite{Treleaven2013}. Because of the secretive nature of AT due to the direct effect of money making, research on specific algorithms often doesn't include specific implementations or pseudo-code of algorithms. This is because they would lose the advantage their particular algorithm has by making it publicly available. 

\section{Algorithmic Stock Trading}

Algorithmic stock trading is a rare combination of being both an academic discipline and a multi-billion dollar industry. Academically, it is often studied in a general context. Specifically, research highlights how AT has a dysfunctional role in the stock market or may take a general look at how HFT can undercut normal trading.  For example, articles review components of AT trading systems, which generally includes data access and cleaning, pre-trade analysis, trading signal generation, trade execution, and post-trade analysis \cite{Treleaven2013}. Other research takes a deep dive into particular experimental algorithms, like \textit{gated Bayesian networks} (GBN) \cite{Bendtsen2016}. GBN's are models that combine several Bayesian networks in a manner that acts like gates -- making specific networks inactive or active based on logic. This method is applied to AT systems and performs better than the benchmark of buy-and-hold while simultaneously substantially decreasing risk of invested capital. 

However, the research also has very practical applications. Numerous textbooks exist that take a very general, overarching dive into algorithmic-trading. Aldridge's textbook \cite{Aldridge2010}, which is one of the most cited works on AT, looks at market structure, HFT data fetching, risk management of HFT, and a variety of na\"{i}ve strategies, such as market making or statistical arbitrage. Gomber et al. \cite{Gomber2011} focus on the evolution of AT and take a look at very specific regulations in European and American markets. These practical applications all reinforce the barriers of entry for an individual trader, as acquiring raw and cleaned data is highly expensive and time consuming due to the time sensitivity of trading algorithms \cite{Treleaven2013}. 

\section{Sentiment Analysis, Twitter, and Stock Trading}

The stock market often behaves as a function of news sentiment. Behavioral finance suggests that emotions, mood, and sentiments in response to news play a significant role in investment \cite{Ho2016}. Particularly, Ho and Wang identify that sentiment has the most significant impact on investors in the market \cite{Ho2016}. Take the GE downswing on April 8, 2019 as an applicable case study. GE's stock plunged over 6\% in pre-market trading as an influential J.P. Morgan analyst slashed his rating on the stock \cite{Sozzi2019}. Significant actors can have major effects on the entire market purely based on sentiment of the actor's remarks. This makes sentiment analysis and stock trading a particular area of interest academically. 


\subsection{Sentiment Analysis Application to Machine Learning and Classification}

Previous literature often uses machine learning for sentiment classification problems. Pang et al. \cite{Pang} employ different machine learning methods on movie review data and find that while it does vastly outperform human-produced baseline, it is not as nearly as effective on traditional topic-based categorization. Their work stems from the issue of automatically classifying the vast amount of online text documents' sentiment, which would be incredibly useful for business intelligence applications and review sites. Using na\"{i}ve Bayes, a maximum entropy model, and support vector machines (SVM), they find that they aren't able to produce very accurate topic-based categorization results. They estimate that a large problem with their sentiment analysis is authors' uses of deliberate contrast in the data, which humans could easily discern unlike machines. Their paper shows how even though machine learning is often relied on for classification and does outperform the baseline, it isn't incredibly effective. These discrepancies could potentially lead to challenges for our own methods. 


\subsection{Past Examinations of Sentiment and Stock Trading}

Using sentiment as an indicator for stock trading has been heavily studied in academia. Li et al. study news impact on stock price using sentiment analysis \cite{Li2014}. Because financial news articles are believed to have an impact on stock price return, they use news sentiment to implement a stock price prediction framework using the Harvard psychological dictionary and the Loughran-McDonald financial sentiment dictionary. This sentiment analysis framework outperforms previous \textit{bag-of-words} stock price prediction frameworks  \cite{Li2014}. This literature demonstrates how news impacts the stock market and can be used to predict stock price movement, which is at the crux of our study. 

Financial markets outside of the major United States heavyweights like the \textit{Dow Jones Industrial Average} (DJIA) have also been studied using sentiment analysis. Garcia et al. use social signals to create a Bitcoin trading algorithm that is highly profitable \cite{Garcia2015}. They integrate various datasources that provide social signal data on information search, word of mouth volume, emotional valence and opinion polarization in the analysis of Bitcoin, a cryptocurrency that is extremely volatile and known for sustained price fluctuations. Their analysis reveals that increases in opinion polarization and exchange volume precede rising Bitcoin prices and are able to create a Bayesian model that is highly profitable. Shah et al. \cite{Shah2014} discuss the efficacy of Bayesian regressions on predicting Bitcoin price variation. They use a latent-source model for the purpose of binary classification, which has been previously established as effective for this specific scenario. They are able to predict price change every 10 seconds accurately enough to double the initial investment in less than 60 days. While these studies consider use of cryptocurrency markets, it still proves that financial instruments are correlated with social signals, or sentiment.  

\subsection{Past Examinations of Twitter as a Valid Means of Stock Trading}

Past literature has extensively examined using Twitter as a means of data for input into trading algorithms. Mao et al. \cite{Mao2013} use Twitter volume spikes to S\&P 500 stocks and wether or not they are useful for stock trading. They find that Twitter volume spikes are correlated with stock price movement, acting as a surprise to market participants based on implied volatility. The authors use a Bayesian classifier to develop a trading strategy that has significant returns in a short period of time. Zhang et al. \cite{Zhang2011} look to predict stock market indicators through Twitter. After collecting randomized twitter posts for six months, the researchers measure collective hope and fear each day and analyze the correlation between those and stock market indicators. They find that emotional tweet percentage is significantly negatively correlated with all of the major United States stock markets. Bollen et al. \cite{Bollen} use Twitter mood to predict the movement of the stock market by classifying specific tweets into six different categories or moods. The authors use a Granger causality analysis and a self-organizing fuzzy neural network to investigate how indicative public mood states are of the DJIA closing prices. They use large-scale Twitter feeds and determine the mood of tweets over 6 dimensions and find that Twitter mood can predict the movement of the DJIA with incredible accuracy. All of these works suggest that Twitter activity can be applied to stock prices and thus used in trading algorithms, which is precisely what we aim to exploit in our study. 

\end{document}

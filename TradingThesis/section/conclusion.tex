\documentclass[../thesis.tex]{subfiles}
\begin{document}
\chapter{Conclusion}
\section{Contributions}
We make 2 different contributions to AT. Our first contribution examines existing AT strategies to understand how they work mathematically to generate buy and sell signals. We then run these different strategies using our test suite to understand the effectiveness of different strategies. We find that combining momentum indicators, specifically RSI and MACD, and pairs trading to be by far and away the most effective. 

Our second part of research looks at using Twitter sentiment as a means of stock trading. We find a lack of statistical correlation between average sentiment and price, motivating our use of different machine learning algorithms. We test two different feature sets for our models, and find that a complex feature set using additional Twitter features such as total retweets to be far more effective than a simplistic feature set. Using this complex feature set, we test the effectivness of each machine learning algorithm, combine multiple signals into a stacking model, and additionally use a deep learning model. We find that our strategies nearly always turn a profit while the baseline measure often posts losses. Even when the stock performs well and the baseline posts profits, our strategies are able to match the performance. 
\section{Future Work}
In the future, our study could be extended greatly. First, the NLP component of the research could be tailored exactly to Twitter. Currently we use the textblob implementation trained on movie review data. Training on tweets would be more applicable to our research and potentially give more accurate sentiment measures. Some of the existing literature looks at applying sentiment classification to cryptocurrencies. Extending this model to other financial markets, such as cryptocurrency, could give differing results. Also, we could extend our models to predict overall market movement. Rather than our models using individual stocks as input, our models could instead use a collection of stocks as input to try to predict DJIA price movement. This mixed effects model for market prediction could be used as an overall portfolio management tool. 
\end{document}
